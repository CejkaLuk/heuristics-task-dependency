\clearpage
\section{Conclusion}
In summary, this protocol presents the theory behind a selection of heuristic methods that can be used to determine the earliest possible completion of a set of jobs while adhering to certain limitations. The limitations include dependencies between the jobs and the resources they require.
The heuristic methods were implemented in a project using Python and subsequently compared on a small exploratory sample of real-life problems observed on a data-processing engine.

Taking into consideration the small sample size, the results presented cannot be used to definitively claim one heuristic outperforms another.
However, it can be stated that in terms of the entire benchmark comprising thousands of problems, PHM and PHMDP outperformed SHM in the majority of cases.
Unfortunately, the specific data cannot be presented due to trade secrets.
It is noteworthy that, theoretically, while SHM produces suboptimal timelines compared to PHM and PHMDP, it may be the preferred heuristic to use as its implementation is faster than the remaining heuristics.
For example, if multiple jobs were to be submitted every second, the timeline may have to be recomputed frequently.
Therefore, it is possible that the time spent computing the timeline may be greater than the time gained by using PHM or PHMDP over SHM.
Furthermore, theoretically, the performance of the implementations of PHM and PHMDP could further decrease with the presence of thousands of jobs.

In terms of future work, the heuristics ought to be compared on a different set of problems.
Furthermore, their implementations ought to be optimized and perhaps implemented using a more performance-oriented language, such as C++.