\section{Introduction}
The development of large-scale projects often requires vast resources, however, the resources available are often limited.
Therefore, completing projects with limited resources before a deadline requires meticulous planning.
Specifically, the resources available at any point in time must be utilized as efficiently as possible.
This goal is one of the key aspects of project management and scheduling.

Many methods that aid in planning the activities of a project with limited resources exist; referred to as \textit{activity dependency problems} in this protocol.
Among the most well-known methods is the Critical Path Method (CPM) \cite{Kelley1959} introduced by \citeauthor{Kelley1959}.
Other methods include heuristics such as those presented in \citetitle{Fiala2008} \cite{Fiala2008}.
While these methods are predominantly used in project management, they can also be used to plan jobs of a data-processing engine.
For this purpose, the protocol aims to introduce a selection of heuristic methods and compare them on a set of problems.

First, the theory behind the heuristic methods is introduced.
Next, the implementation done as part of the assessment project is briefly described.
Then, the results of the comparison of the heuristic methods across several problems is presented.
Finally, the last section summarizes the contents of this protocol.
