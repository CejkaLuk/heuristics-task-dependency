\documentclass[czech, american]{article}

%%% Packages
%% Encoding, paper settings, etc.
\usepackage[T1]{fontenc}			% Font encoding
\usepackage[utf8]{inputenc}			% Input encoding
\usepackage[a4paper]{geometry}		% Use of the A4 page

%% Languages
\usepackage[czech, american]{babel}							 % Czech and American English are used in this document
\usepackage{silence}										 % Silence output of warnings
\WarningFilter{biblatex}{File 'american-iso.lbx' not found!} % Silence weird message that I couldn't find a solution for

%% Fonts
\usepackage{amsmath}				% Advanced mathematics package
\usepackage{csquotes}				% Ensure that quoted texts are typeset according to the rules of the main language
\usepackage{soul}					% Text fonts syllable by syllable - used for in-text codifying
\usepackage[hidelinks]{hyperref}	% Hide border around URL links
\hypersetup{
	colorlinks=true, citecolor=blue, filecolor=blue, linkcolor=blue,
	urlcolor=blue
}

%% Graphics
\usepackage{tikz}					% Visual representation of matrices, graphs etc.
\usetikzlibrary{
	external,							% Externalize tikz pictures to avoid recompiling them with every compilation
	positioning,
	arrows.meta
}
\tikzexternalize[					% Add '-shell-escape' to the pdflatex compilation command. In TeXstudio: Options -> Configure Texstudio... -> Commands -> PdfLaTeX: 'pdflatex --synctex=1 -interaction=nonstopmode -shell-escape %.tex'
prefix=images/tikz/,				% Place externalized TikZ-generated images in a specific folder - ignored by git
]
\usepackage{pgfplots}				% For visualizations, graphics, etc.
\pgfplotsset{compat=1.3}			% Use pgfplots version 1.3 specifically

%% Code:
\usepackage{listings} 				% Code formatting package

%% Referencing
\usepackage[
	backend=biber,								  		  	% In TeXstudio, the backend here must correspond to: Options -> Configure TeXstudio... -> Build -> Default Bibliography Tool
	style=iso-numeric,
	sorting=none,
	giveninits=true
]{biblatex}								% Bibliography: ISO 690 format, cite shows numbers, sort according to occurrence in the text, shorten first names
\addbibresource{contents/bibliography.bib}				  	% Specify path to bibliography file
\DeclareFieldFormat{labelnumberwidth}{\mkbibbrackets{#1}}	% Square brackets around numbers in the bibliography
\DeclareFieldFormat*{citetitle}{\mkbibemph{#1}}				% \citetitle{bibid} in text produces the title in emphasis for all sources. Without this, for example, the 'thesis' citation entry is non-emphasized with quotes while online sources are only emphasized (looks bad)
\setcounter{biburllcpenalty}{7000}						  	% Insert breakpoints after lowercase letters in URLs in the bibliography
\setcounter{biburlucpenalty}{8000}						  	% Insert breakpoints after uppercase letters in URLs in the bibliography
\DeclareCiteCommand{\citeauthor}							% Custom format of citeauthor: 'J. H. Watson`
{\boolfalse{citetracker}%
	\boolfalse{pagetracker}%
	\usebibmacro{prenote}}
{\ifciteindex
	{\indexnames{labelname}}
	{}%
	\printnames[given-family]{labelname}}
{\multicitedelim}
{\usebibmacro{postnote}}
\usepackage{afterpage}										% To keep the footnote created in the caption of a figure and the figure itself on the same page

%% Document settings
\parindent=0pt 		   % Indentation of the 1st line of the paragraph
\parskip=7pt   		   % Space between paragraphs

%%% Custom colors
\definecolor{gray-light}{gray}{0.95}			% Inline code highlighting color
\definecolor{lbcolor}{rgb}{0.9,0.9,0.9}			% Code block background color

%%% Custom code block
\lstset{
	backgroundcolor=\color{lbcolor}, % Background color of the code block
	upquote=true,					 % Format of quote: true -> '', false -> ‘’
	columns=fixed,					 % Characters are below each other (each column is one character -> fixed column size)
	extendedchars=false,			 % Allow national characters, for example, Czech diacritics. If true, then load any package that defines the characters, for example, fontenc, or inputenc, etc.
	showtabs=false,					 % Tabulators visible or not
	showspaces=false,				 % All blank spaces visible as _ or not
	showstringspaces=false,			 % Blank spaces in strings visible as _ or not 
	identifierstyle=\ttfamily,		 % Code font to be monospace if the line begins with one of the following: a-z, A-Z, @, $ and _
	language=Python,				 % Default language syntax highlighting for code blocks
	captionpos=b,					 % Position of the caption. b - bottom; t - top of the listing
	tabsize=2,						 % Set tabulator stops			
	frame=lines,					 % Draw a line on the top and bottom of the code listing -> frame
	numbers=left,					 % Print line numbers on the left of the code block
	numberstyle=\tiny,				 % Font and size of line numbers
	numbersep=5pt,					 % Distance between line numbers and the code block
	basicstyle=\footnotesize,		 % Basic font. Selected at the beginning of each listing
	keywordstyle=\color[rgb]{0,0,1}, % Fond of language keywords in a specific font
	commentstyle=\color{green-dark}, % Font of comments in code block
	stringstyle=\color{red},		 % Font of non-keywords, comments and strings
	breaklines=true,				 % Breaking of long lines
	prebreak = \raisebox{0ex}[0ex][0ex]{\ensuremath{\hookleftarrow}}, % Line-break symbol if a line of code is too long
}

%%% Custom commands
\newcommand{\code}[1]{\colorbox{gray-light}{\texttt{#1}}}									   % Inline code font/highlighting
\newenvironment{tight_enumerate}{															   % Define new enumerate with smaller spaces between items
	\begin{enumerate}
		\setlength{\itemsep}{0pt}
	}{\end{enumerate}}
\newenvironment{tight_itemize}{															  	   % Define a new itemize environment with smaller spaces between items
	\begin{itemize}
		\setlength{\itemsep}{0pt}
	}{\end{itemize}}

%%% String variables
\newcommand{\Author}{Lukáš Matthew Čejka}
\newcommand{\Institute}{FNSPE CTU in Prague}
\newcommand{\Date}{\today}
\newcommand{\ProtocolSubTitle}{Heuristic Algorithms - Assessment Task Protocol}
\newcommand{\ProtocolTitle}{Analysis of Heuristics Methods for Acitivity Dependency Problems}

\begin{document}

\title{\ProtocolSubTitle \\
\textbf{\ProtocolTitle}}
\author{\Author}
\maketitle

%%%%%%%%%%%% CONTENTS OF THE PAPER %%%%%%%%%%%%
\selectlanguage{american}
{
	\hypersetup{linkcolor=black}
	\tableofcontents
}

%--------------------------------------------------------
%|         The PAPER ITSELF begins here                 |
%--------------------------------------------------------

\newpage

\section{Introduction}
The development of large-scale projects often requires vast resources, however, the resources available are often limited.
Therefore, completing projects with limited resources before a deadline requires meticulous planning.
Specifically, the resources available at any point in time must be utilized as efficiently as possible.
This goal is one of the key aspects of project management and scheduling.

Many methods that aid in planning the activities of a project with limited resources exist; referred to as \textit{activity dependency problems} in this protocol.
Among the most well-known methods is the Critical Path Method (CPM) \cite{Kelley1959} introduced by \citeauthor{Kelley1959}.
Other methods include heuristics such as those presented in \citetitle{Fiala2008} \cite{Fiala2008}.
While these methods are predominantly used in project management, they can also be used to plan jobs of a data-processing engine.
For this purpose, the protocol aims to introduce a selection of heuristic methods and compare them on a set of problems.

First, the theory behind the heuristic methods is introduced.
Next, the implementation done as part of the assessment project is briefly described.
Then, the results of the comparison of the heuristic methods across several problems is presented.
Finally, the last section summarizes the contents of this protocol.


\section{Theory}
To facilitate a better understanding of the heuristic methods presented in this protocol, key concepts must first be introduced.

\paragraph{Problem}
In the context of this protocol, a \textit{problem} represents a project that is made up of activities.
Each activity takes a certain amount of time to complete and a set amount of resources at each point in time.
To complete the project, all activities must be finished while adhering to their dependencies and the resources available for the project at each point in time.
The challenging aspect of the problem is to complete the project in the shortest time possible while keeping to the dependencies, and not exceeding the resources available. The activities and their relationships within a project can be represented using an event-oriented directed graph, called the \textit{CPM network}.

In a CPM network, the nodes represent milestones that mark the start or completion of activities, while the directed, weighted edges represent the activities and their duration.
For this project, it is assumed that there is only one start node initiating the project and one end node that represents the completion of the project.
An example of such a graph is presented in Figure~\ref{Figure:theory->problem->cpm-network-example}.

\begin{figure}[ht!]
	\centering
	\begin{tikzpicture}[
		mynode/.style={
			circle,
			draw=black,
			fill=gray,
			fill opacity = 0.3,
			text opacity=1,
			inner sep=0pt,
			minimum size=20pt,
			font=\small},
		myarrow/.style={-Stealth},
		node distance=0.6cm and 1.2cm
		]
		\node[mynode] (n1) {1};
		\node[mynode,above right=of n1] (n2) {2};
		\node[mynode,below=of n2] (n3) {3};
		\node[mynode,below=of n3] (n4) {4};
		\node[mynode,right=of n3] (n5) {5};
		\node[mynode,right=of n5] (n6) {6};
		
		\foreach \i/\j/\txt/\p in {% start node/end node/text/position
			n1/n2/4/above,
			n1/n3/3/above,
			n1/n4/5/above,
			n2/n5/3/above,
			n3/n5/2/above,
			n4/n6/4/below,
			n5/n6/3/above}
			\draw [myarrow] (\i) -- node[sloped,font=\small,\p] {\txt} (\j);
			
	\end{tikzpicture}
	\caption{Visualization of a project comprising activities as a CPM network.
		Each node represents a milestone that either marks the start or the completion of activities.
		The number inside each node is the ID of the milestone represented by the node.
		Each weighted edge represents an activity and its duration.
		The ID of an activity is dependent on the node it starts from and the node it ends in, i.e. $i$-$j$, where $i$ is the ID of the start node and $j$ is the ID of the end node.
		For example, the ID of the activity starting in node 1 and ending in node 2 is 1-2 and its duration is 4.
	}
	\label{Figure:theory->problem->cpm-network-example}
\end{figure}

For clarity, the term \textit{dependency of activities} means that an activity cannot be started before all of its predecessors have been completed. For example, in Figure~\ref{Figure:theory->problem->cpm-network-example}, activity 5-6 can only be started once activities 2-5 and 3-5 have been completed.

The graph presented in Figure~\ref{Figure:theory->problem->cpm-network-example} is referred to as a CPM network due to its widespread use during the analysis of projects using the Critical Path Method (CPM) \cite{Kelley1959}.

\paragraph{Critical Path Method (CPM)}
The Critical Path Method is an algorithm used for scheduling activities of a project.
It is used to determine the earliest possible end of the project while only adhering to the dependencies of activities and their durations, i.e., not their resources.
Given its dependencies and duration ($t_{i, j}$), for each activity, the algorithm of CPM determines the following: 

\begin{tight_itemize}
	\item Earliest Start (ES) - Earliest possible time the activity can \textit{start} considering its dependencies.
	\item Earliest End (EE) - Earliest possible time the activity can \textit{end} considering its dependencies.
	\item Latest Start (LS) - Latest permissible time the activity can \textit{start} considering its dependencies.
	\item Latest End (LE) - Latest permissible time the activity can \textit{end} considering its dependencies.
	\item Time Reserves (TR) - Number of time units that the activity can be delayed before the end of the project must be delayed.
		It is equal to the difference between LE and EE (or LS and ES).
\end{tight_itemize}

If the time reserves for an activity are equal to zero, then delaying the activity means delaying the project.
Activities with zero time reserves are referred to as \textit{critical activities} and they make up the so-called \textit{critical path}.
The critical path is a sequence of dependent activities between the start and end nodes that dictates when the project can be completed earliest.

For example, the output of CPM for the project shown in Figure~\ref{Figure:theory->problem->cpm-network-example} is presented in Table~\ref{Table:theory->cpm->exmaple->data}.

\begin{table}[ht]
	\centering
	\begin{tabular}{|c|c|c|c|c|c|c|}
		\hline
		$i$-$j$ & $t_{i, j}$ & ES$_{i, j}$ & EE$_{i, j}$ & LS$_{i, j}$ & LE$_{i, j}$ & TR$_{i, j}$ \\ \hline
		  1-2   &     4      &      0      &      4      &      0      &      4      &      0      \\
		  1-3   &     3      &      0      &      3      &      2      &      5      &      2      \\
		  1-4   &     5      &      0      &      5      &      1      &      6      &      1      \\
		  2-5   &     3      &      4      &      7      &      4      &      7      &      0      \\
		  3-5   &     2      &      3      &      5      &      5      &      7      &      2      \\
		  4-6   &     4      &      5      &      9      &      6      &     10      &      1      \\
		  5-6   &     3      &      7      &     10      &      7      &     10      &      0      \\ \hline
	\end{tabular}
	\caption{Output of CPM for the project presented in Figure~\ref{Figure:theory->problem->cpm-network-example}.
		The critical path is made up of activities 1-2, 2-5, and 5-6 since their TRs are equal to zero.
		The last activities, 4-6 and 5-6, are both completed at time unit 10, therefore, the entire project can be completed in 10 time units.
	}
	\label{Table:theory->cpm->exmaple->data}
\end{table}

The activity timeline produced by CPM can be visualized using a Gantt chart\footnote{Gantt chart Wikipedia URL: \url{https://en.wikipedia.org/wiki/Gantt_chart}}, as shown in Figure~\ref{Figure:theory->cpm->example->timeline}.

\afterpage{
\begin{figure}[ht!]
	\centering
	\includegraphics[width=\linewidth]{images/cpm_example_project.png}
	\caption{Timeline of the activities presented in Table~\ref{Table:theory->cpm->exmaple->data} as scheduled by CPM.
		The vertical axis represents the activity, while the horizontal axis displays the time units.
		The numbers in the squares represent the units of resources required by each activity at a point in time.
		Note that CPM does not take the resources into account, they are only displayed here for completeness.
		The image was generated using the Matplotlib Python package\protect\footnotemark[2].
	}
	\label{Figure:theory->cpm->example->timeline}
\end{figure}
\footnotetext[2]{Matplotlib webpage URL: \url{https://matplotlib.org/stable/index.html}}
}

The algorithm of CPM is omitted as the output of CPM is the input for the heuristic methods that schedule the activities of a project while adhering to the dependencies, the durations, and the resources.

\subsection{Heuristic Methods}
As mentioned earlier, CPM determines the earliest possible end of a project while taking into account only the dependencies of activities and their durations.
On the other hand, the heuristic methods analyzed in this protocol also take into account the resources required by each activity at a point in time.

The heuristic methods selected from from \citetitle{Fiala2008} \cite{Fiala2008} for this assessment project are the following:

\begin{tight_itemize}
	\item Serial Heuristic Method (SHM)
	\item Parallel Heuristic Method (PHM)
	\item Parallel Heuristic Method with Dynamic Priorities (PHMDP)
\end{tight_itemize}

All of the above-mentioned heuristic methods use the output of CPM as their input.
Additionally, the heuristic methods take in the $r_\mathrm{max}$ parameter which represents the maximum number of resources available at each point in time.



\subsection{Serial Heuristic Method (SHM)}
Using the output of CPM and the provided $r_\mathrm{max}$, the Serial Heuristic Method performs the following steps:

\begin{tight_enumerate}
	\item Arrange the activities into a sequence $\left(i_1, j_1\right),  \ldots, \left(i_m, j_m\right)$ so that $i_k < i_{k+1}, j_k < j_{k+1}$.
	\item Starting with the first activity, schedule each activity while adhering to the dependencies and the maximum resources available at a point in time ($r_\mathrm{max}$).
\end{tight_enumerate}

To demonstrate the process, the activity timeline for the example project scheduled using SHM with $r_\mathrm{max} = 6$ is shown in Figure~\ref{Figure:theory->shm->example->timeline}.

\begin{figure}[ht!]
	\centering
	\includegraphics[width=\linewidth]{images/shm_example_project.png}
	\caption{Timeline of the activities presented in Table~\ref{Table:theory->cpm->exmaple->data} as scheduled by SHM.
		The vertical axis represents the activity, while the horizontal axis displays the time units.
		The numbers in the squares represent the units of resources required by each activity at a point in time.
	}
	\label{Figure:theory->shm->example->timeline}
\end{figure}

Specifically, the order of activities in the sequence is identical to the order of activities in column $\left(i, j\right)$ in Table~\ref{Table:theory->cpm->exmaple->data}.
SHM schedules activity 1-2 from time 0.
Then it tries to schedule activity 1-3, however, it can only do so from time 4 when activity 1-2 has finished as the resources required by both activities would exceed the resources available $3 + 4 = 7 > 6$.
According to SHM, the project can be completed in 16 time units while adhering to both dependencies and the available resources.

From Figure~\ref{Figure:theory->shm->example->timeline}, it can be seen that SHM is noticeably inefficient.
For example, from time 0, activities 1-2 and 1-4 can be scheduled while not exceeding the available resources.
However, since activity 1-3 is scheduled earlier, activity 1-4 cannot be scheduled until time unit 7.
This inefficiency is addressed by the \textit{Parallel Heuristic Method}.



\subsection{Parallel Heuristic Method (PHM)}
Unlike SHM, the Parallel Heuristic Method (PHM) provides each activity with a priority equal to its Time Reserves (TR) value.
Counterintuitively, the lower the priority value of an activity, the higher the priority of the activity.
This is due to TR representing the number of time units that an activity can be delayed for.
Thus, if $\mathrm{TR_{i ,j}} = 0$, then activity $i$-$j$ must be scheduled as soon as possible, otherwise the project may be delayed.

Using the output of CPM and the provided $r_\mathrm{max}$, PHM performs the following steps:

\begin{tight_enumerate}
	\item For every time unit:
	\begin{tight_enumerate}
		\item Create a set of activities whose predecessors are all completed.
		\item Arrange the activities in the set in ascending order according to their TR value.
		\item Starting with the activity with the lowest TR value, schedule as many activities from the time unit as possible while adhering to the resources available in every time unit.
	\end{tight_enumerate}
\end{tight_enumerate}

The activity timeline for the example project scheduled using PHM with $r_\mathrm{max} = 6$ is shown in Figure~\ref{Figure:theory->phm->example->timeline}.

\begin{figure}[ht!]
	\centering
	\includegraphics[width=\linewidth]{images/phm_example_project.png}
	\caption{Timeline of the activities presented in Table~\ref{Table:theory->cpm->exmaple->data} as scheduled by PHM.
		The vertical axis represents the activity, while the horizontal axis displays the time units.
		The numbers in the squares represent the units of resources required by each activity at a point in time.
	}
	\label{Figure:theory->phm->example->timeline}
\end{figure}

As can be seen from Figure~\ref{Figure:theory->phm->example->timeline}, unlike SHM, PHM scheduled activities 1-2 and 1-4 to run parallel.
The prioritization of activities allows PHM to determine that the project can be completed in 15 time units while adhering to both the dependencies and available resources.

However, PHM has a weakness: statically set priorities.
The priorities set by PHM do not change with the flow of time.
Therefore, the completion of non-critical activities can be delayed, thus causing suboptimal scheduling.
For example, since the TR value of activity 1-3 is 2, its scheduling is delayed until time 7.
Therefore, activities 3-5 and 5-6 are not even considered until time 10 which delays the entire project.
This flaw is addressed using the \textit{Parallel Heuristic Method with Dynamic Priorities}.



\subsection{Parallel Heuristic Method with Dynamic Priorities (PHMDP)}
The Parallel Heuristic Method with Dynamic Priorities (PHMDP) sets the priority value of each activity to $\mathrm{LS}_{i, j} - t$, where $t$ represents the time.
Furthermore, for every $t$, PHMDP updates the priorities, which eliminates the inadequacy of PHM.

Using the output of CPM and the provided $r_\mathrm{max}$, PHMDP performs the following steps:

\begin{tight_enumerate}
	\item For every time unit $t$:
	\begin{tight_enumerate}
		\item Create a set of activities whose predecessors are all completed.
		\item Update the priorities for the activities to $\mathrm{LS}_{i, j} - t$.
		\item Arrange the activities in the set in ascending order according to their priority value.
		\item Starting with the activity with the lowest priority value, schedule as many activities from the time unit as possible while adhering to the resources available in every time unit.
	\end{tight_enumerate}
\end{tight_enumerate}

The activity timeline for the example project scheduled using PHMDP with $r_\mathrm{max} = 6$ is shown in Figure~\ref{Figure:theory->phmdp->example->timeline}.

\begin{figure}[ht!]
	\centering
	\includegraphics[width=\linewidth]{images/phmdp_example_project.png}
	\caption{Timeline of the activities presented in Table~\ref{Table:theory->cpm->exmaple->data} as scheduled by PHMDP.
		The vertical axis represents the activity, while the horizontal axis displays the time units.
		The numbers in the squares represent the units of resources required by each activity at a point in time.
	}
	\label{Figure:theory->phmdp->example->timeline}
\end{figure}

As shown in Figure~\ref{Figure:theory->phmdp->example->timeline} the dynamic priorities allow for activities to be scheduled more efficiently.
Ultimately, this results in PHMDP determining that the project can be completed in 13 time units while adhering to both the dependencies and available resources.

To examine the behavior of the methods further, they were implemented and compared on more examples.



\setcounter{footnote}{3} % Workaround for using footnotemark in theory.tex
\section{Implementation}
In this section, the implementation of the project comprising the methods introduced in Section~\ref{Section:theory} is presented.
The source code is available on request or in the project's GitHub repository\footnote{Heuristic Methods for Activity Dependency Problems GitHub repository URL: \url{https://github.com/CejkaLuk/heuristics-task-dependency}}.
The project was implemented in Python (version 3.9.6\footnote{Python 3.9.6 available at: \url{https://www.python.org/downloads/release/python-396}}) as it offers clean data structures and support for visualization tools.

The project uses the following \textit{make}\footnote{GNU Make webpage URL: \url{https://www.gnu.org/software/make}} tasks:

\begin{tight_itemize}
	\item \code{make init} - Download and install the required Python packages.
	\item \code{make tests}, \code{make tests\_coverage}, and \code{make tests\_coverage\_report} - Run the unit tests using \textit{nose2}\footnote{Nose2 testing framework webpage URL: \url{https://docs.nose2.io/en/latest}} alone, with coverage, and with coverage generated into an HTML report, respectively.
	\item \code{make docs} (executed in \code{docs/}) - Generate the project documentation using \textit{Sphinx}\footnote{Sphinx documentation generator webpage URL: \url{https://www.sphinx-doc.org/en/master}}.
	\item \code{make clean} (executed in the repository root or in \code{docs/}) - Clean the generated files.
\end{tight_itemize}

The core functions of SHM, PHM, and PHMDP are presented in Listings~\ref{Listing:implementation->shm->solve-function}, \ref{Listing:implementation->phm->solve-function}, and \ref{Listing:implementation->phmdp->solve-function}, respectively.

\begin{lstlisting}[caption={The core functions of SHM: \code{solve()} and \code{\_schedule\_activity()}. Note that unimportant functions have been omitted for brevity.},label={Listing:implementation->shm->solve-function}]
def solve(self):
"""Solves the activity dependency problem with resources."""
	
	self.cpm.solve()
	
	# Schedule activities
	for act in self.cpm.project.activities:
		self._schedule_activity(act)
	
	self.cpm.project.actual_end = self._get_project_actual_end()
	
def _schedule_activity(self, act: Activity):
	"""Schedules an activity as soon as possible considering dependencies and available resources."""

	# Get the time when all precessors of act have been completed
	time = self._get_predecessors_finished_time(act)
	
	while not act.is_scheduled():
		tentative_act_end = time + act.duration
		
		# Get the time when the available resources are exceeded between 'time' and 'tentative_act_end'
		time_resources_exceed = self._get_time_available_resources_exceeded(act, time, tentative_act_end)
		
		# If the resources are not exceeded in the time frame, then schedule the activity from 'time'
		if time_resources_exceed is None:
			self._schedule_activity_from(act, time)
		# Otherwise proceed to the next time when the resources are not exceeded
		else:
			time = time_resources_exceed + 1
\end{lstlisting}

\begin{lstlisting}[caption={The core function of PHM: \code{solve()}.},label={Listing:implementation->phm->solve-function}]
def solve(self):
	"""Solves the activity dependency problem with resources and time reserves as priorities."""
	
	self.cpm.solve()
	
	self._init_activity_priorities()
	
	time = 0
	while self._unfinished_activities_exist(time):
		# For PHM, this function does nothing, it serves as a placeholder so that PHMDP can reuse the 'solve' function
		self._update_priorities(time)
		
		# Get activities that can be schedule from 'time'
		viable_activities = self._get_viable_activities(time)
		
		if len(viable_activities) > 0:
			self._sort_by_priority_and_id(viable_activities)
			
			# Try to schedule all viable activities if they don't exceed the available resources
			for act in viable_activities:
				if not self._resources_exceeded(act, time, time + act.duration):
					self._schedule_activity_from(act, time)
		
		# Jump to the next time when an activity finishes as that is when more activities can be scheduled
		time = self._get_time_next_act_finish(time)
	
	self.cpm.project.actual_end = self._get_project_actual_end()
\end{lstlisting}

\begin{lstlisting}[caption={The core functions of PHMDP. The PHMDP class inherits from PHM, therefore, it only overrides functions that deal with priorities.},label={Listing:implementation->phmdp->solve-function}]
def _init_activity_priorities(self):
	"""Override the method in PHM to avoid initializing activities without time."""

def _update_priorities(self, time: int):
	"""Override the method in PHM to update the priorities dynamically."""
	for act in self.cpm.project.activities:
		act.priority = act.latest_start - time
\end{lstlisting}

\section{Comparison}
This section presents the results of the comparison of the heuristic methods presented in Section~\ref{Section:theory}.
While the heuristics were compared on thousands of problems, only the results of four are presented in this protocol due to trade secrets.
Specifically, the methods were compared across four real-life scheduling problems obtained from a data-processing engine.
The problems contained between 7 and 12 activities, and the resources corresponded to the number of CPU cores available to the engine.
One problem was presented in Section~\ref{Section:theory}, and three problems are presented in this section.

For each problem, the CPM network is presented.
In this section, the edges representing jobs in the CPM networks have two values associated with them: computation time in seconds and CPU cores required.

\subsection{Problem 1}
The visualization of problem 1 as a CPM network is shown in Figure~\ref{Figure:comparion->problem->1->cpm-network}. For this problem, $r_\mathrm{max}$ was set to 7.

\begin{figure}[ht!]
	\centering
	\begin{tikzpicture}[
		mynode/.style={
			circle,
			draw=black,
			fill=gray,
			fill opacity = 0.3,
			text opacity=1,
			inner sep=0pt,
			minimum size=20pt,
			font=\small},
		myarrow/.style={-Stealth},
		node distance=0.6cm and 1.2cm
		]
		\node[mynode] (n1) {1};
		\node[mynode,above right=of n1] (n2) {2};
		\node[mynode,below=of n2] (n3) {3};
		\node[mynode,below=of n3] (n4) {4};
		\node[mynode,right=of n3] (n5) {5};
		\node[mynode,right=of n5] (n6) {6};
		
		\foreach \i/\j/\txt/\p in {% start node/end node/text/position
			n1/n2/{4, 3}/above,
			n1/n3/{6, 5}/above,
			n1/n4/{5, 4}/above,
			n2/n5/{3, 3}/above,
			n3/n5/{4, 3}/above,
			n4/n6/{4, 5}/below,
			n5/n6/{3, 3}/above}
		\draw [myarrow] (\i) -- node[sloped,font=\small,\p] {\txt} (\j);
		
	\end{tikzpicture}
	\caption{Visualization of problem 1 as a CPM network.
		Each weighted edge represents a job, its duration (in seconds), and the number of CPU cores required.
	}
	\label{Figure:comparion->problem->1->cpm-network}
\end{figure}

The timelines produced by SHM, PHM, and PHMDP are shown in Figures~\ref{Figure:comparison->problem->1->timelines->shm}, \ref{Figure:comparison->problem->1->timelines->phm}, and \ref{Figure:comparison->problem->1->timelines->phmdp}, respectively.
As can be seen from the figure, unexpectedly, PHM produced the best results for this problem as it scheduled the jobs to be completed within 20 seconds.
However, PHMDP was only one second behind.
This is due to the dynamic priorities holding back certain jobs, such as 1-4, even though they need not have been.

\begin{figure}[ht!]
	\centering
	\includegraphics[width=0.7\linewidth]{images/comparison/shm_problem_1.png}
	\caption{Timeline of the activities in problem 1 as scheduled by SHM.
		The vertical axis represents the job, while the horizontal axis displays the time in seconds.
		The numbers in the squares represent the CPU cores required by each job at a point in time.
	}
	\label{Figure:comparison->problem->1->timelines->shm}
\end{figure}

\begin{figure}[ht!]
	\centering
	\includegraphics[width=0.7\linewidth]{images/comparison/phm_problem_1.png}
	\caption{Timeline of the activities in problem 1 as scheduled by PHM.}
	\label{Figure:comparison->problem->1->timelines->phm}
\end{figure}

\begin{figure}[ht!]
	\centering
	\includegraphics[width=0.7\linewidth]{images/comparison/phmdp_problem_1.png}
	\caption{Timeline of the activities in problem 1 as scheduled by PHMDP.}
	\label{Figure:comparison->problem->1->timelines->phmdp}
\end{figure}

\subsection{Problem 2}
The visualization of problem 2 as a CPM network is shown in Figure~\ref{Figure:comparion->problem->2->cpm-network}. For this problem, $r_\mathrm{max}$ was set to 8.

\begin{figure}[ht!]
	\centering
	\begin{tikzpicture}[
		mynode/.style={
			circle,
			draw=black,
			fill=gray,
			fill opacity = 0.3,
			text opacity=1,
			inner sep=0pt,
			minimum size=20pt,
			font=\small},
		myarrow/.style={-Stealth},
		node distance=0.6cm and 1.2cm
		]
		\node[mynode] (n1) {1};
		\node[mynode,above right=of n1] (n2) {2};
		\node[mynode,below right=of n1] (n3) {3};
		\node[mynode,below right=of n2] (n4) {4};
		\node[mynode,above right=of n4] (n5) {5};
		\node[mynode,below right=of n4] (n6) {6};
		\node[mynode,below right=of n5] (n7) {7};
		
		\foreach \i/\j/\txt/\p in {% start node/end node/text/position
			n1/n2/{5, 3}/above,
			n1/n3/{6, 4}/above,
			n1/n4/{10, 3}/above,
			n2/n4/{1, 3}/above,
			n2/n5/{6, 2}/above,
			n3/n4/{2, 3}/above,
			n3/n6/{5, 3}/above,
			n4/n5/{8, 3}/below,
			n4/n6/{7, 2}/below,
			n5/n6/{9, 4}/above,
			n5/n7/{7, 4}/above,
			n6/n7/{12, 4}/above}
		\draw [myarrow] (\i) -- node[sloped,font=\small,\p] {\txt} (\j);
		
	\end{tikzpicture}
	\caption{Visualization of problem 2 as a CPM network.
		Each weighted edge represents a job, its duration (in seconds), and the number of CPU cores required.
	}
	\label{Figure:comparion->problem->2->cpm-network}
\end{figure}

The timelines produced by SHM, PHM, and PHMDP are shown in Figures~\ref{Figure:comparison->problem->2->timelines->shm}, \ref{Figure:comparison->problem->2->timelines->phm}, and \ref{Figure:comparison->problem->2->timelines->phmdp}, respectively.
In the case of problem 2, PHMDP scheduled the jobs to be completed in the shortest time: 41 seconds.
However, the slowest heuristic, SHM, was only three seconds behind.

\begin{figure}[ht!]
	\centering
	\includegraphics[width=0.8\linewidth]{images/comparison/shm_problem_2.png}
	\caption{Timeline of the activities in problem 2 as scheduled by SHM.
		The vertical axis represents the job, while the horizontal axis displays the time in seconds.
		The numbers in the squares represent the CPU cores required by each job at a point in time.
	}
	\label{Figure:comparison->problem->2->timelines->shm}
\end{figure}

\begin{figure}[ht!]
	\centering
	\includegraphics[width=0.8\linewidth]{images/comparison/phm_problem_2.png}
	\caption{Timeline of the activities in problem 2 as scheduled by PHM.}
	\label{Figure:comparison->problem->2->timelines->phm}
\end{figure}

\begin{figure}[ht!]
	\centering
	\includegraphics[width=0.8\linewidth]{images/comparison/phmdp_problem_2.png}
	\caption{Timeline of the activities in problem 2 as scheduled by PHMDP.}
	\label{Figure:comparison->problem->2->timelines->phmdp}
\end{figure}

\subsection{Problem 3}
The visualization of problem 3 as a CPM network is shown in Figure~\ref{Figure:comparion->problem->3->cpm-network}. For this problem, $r_\mathrm{max}$ was set to 6.

\begin{figure}[ht!]
	\centering
	\begin{tikzpicture}[
		mynode/.style={
			circle,
			draw=black,
			fill=gray,
			fill opacity = 0.3,
			text opacity=1,
			inner sep=0pt,
			minimum size=20pt,
			font=\small},
		myarrow/.style={-Stealth},
		node distance=0.6cm and 1.2cm
		]
		\node[mynode] (n1) {1};
		\node[mynode,above right=of n1] (n2) {2};
		\node[mynode,below=of n2] (n3) {3};
		\node[mynode,below=of n3] (n4) {4};
		\node[mynode,right=of n2] (n5) {5};
		\node[mynode,right=of n5] (n7) {7};
		\node[mynode,below=of n7] (n6) {6};
		
		
		\foreach \i/\j/\txt/\p in {% start node/end node/text/position
			n1/n2/{4, 3}/above,
			n1/n3/{3, 4}/above,
			n1/n4/{5, 3}/above,
			n2/n5/{3, 2}/above,
			n3/n5/{2, 3}/above,
			n4/n6/{4, 2}/below,
			n5/n6/{3, 4}/above,
			n5/n7/{4, 3}/above}
		\draw [myarrow] (\i) -- node[sloped,font=\small,\p] {\txt} (\j);
		
	\end{tikzpicture}
	\caption{Visualization of problem 3 as a CPM network.
		Each weighted edge represents a job, its duration (in seconds), and the number of CPU cores required.
	}
	\label{Figure:comparion->problem->3->cpm-network}
\end{figure}

The timelines produced by SHM, PHM, and PHMDP are shown in Figures~\ref{Figure:comparison->problem->3->timelines->shm}, \ref{Figure:comparison->problem->3->timelines->phm}, and \ref{Figure:comparison->problem->3->timelines->phmdp}, respectively.
In the case of problem 3, PHM and PHDMP tied in scheduling the jobs to complete in the shortest time: 17 seconds.
However, SHM was only two seconds behind.

\begin{figure}[ht!]
	\centering
	\includegraphics[width=0.8\linewidth]{images/comparison/shm_problem_3.png}
	\caption{Timeline of the activities in problem 3 as scheduled by SHM.
		The vertical axis represents the job, while the horizontal axis displays the time in seconds.
		The numbers in the squares represent the CPU cores required by each job at a point in time.
	}
	\label{Figure:comparison->problem->3->timelines->shm}
\end{figure}

\begin{figure}[ht!]
	\centering
	\includegraphics[width=0.8\linewidth]{images/comparison/phm_problem_3.png}
	\caption{Timeline of the activities in problem 3 as scheduled by PHM.}
	\label{Figure:comparison->problem->3->timelines->phm}
\end{figure}

\begin{figure}[ht!]
	\centering
	\includegraphics[width=0.8\linewidth]{images/comparison/phmdp_problem_3.png}
	\caption{Timeline of the activities in problem 3 as scheduled by PHMDP.}
	\label{Figure:comparison->problem->3->timelines->phmdp}
\end{figure}

\clearpage
\section{Conclusion}
In summary, this protocol presents the theory behind a selection of heuristic methods that can be used to determine the earliest possible completion of a set of jobs while adhering to certain limitations. The limitations include dependencies between the jobs and the resources they require.
The heuristic methods were implemented in a project using Python and subsequently compared on a small exploratory sample of real-life problems observed on a data-processing engine.

Taking into consideration the small sample size, the results presented cannot be used to definitively claim one heuristic outperforms another.
However, it can be stated that in terms of the entire benchmark comprising thousands of problems, PHM and PHMDP outperformed SHM in the majority of cases.
Unfortunately, the specific data cannot be presented due to trade secrets.
It is noteworthy that, theoretically, while SHM produces suboptimal timelines compared to PHM and PHMDP, it may be the preferred heuristic to use as its implementation is faster than the remaining heuristics.
For example, if multiple jobs were to be submitted every second, the timeline may have to be recomputed frequently.
Therefore, it is possible that the time spent computing the timeline may be greater than the time gained by using PHM or PHMDP over SHM.
Furthermore, theoretically, the performance of the implementations of PHM and PHMDP could further decrease with the presence of thousands of jobs.

In terms of future work, the heuristics ought to be compared on a different set of problems.
Furthermore, their implementations ought to be optimized and perhaps implemented using a more performance-oriented language, such as C++.

\printbibliography

\nocite{*}

\end{document}